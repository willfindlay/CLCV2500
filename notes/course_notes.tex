% Options for packages loaded elsewhere
\PassOptionsToPackage{unicode}{hyperref}
\PassOptionsToPackage{hyphens}{url}
\PassOptionsToPackage{dvipsnames,svgnames*,x11names*}{xcolor}
%
\documentclass[
  12pt]{findlay}
\usepackage{lmodern}
\usepackage{amssymb,amsmath}
\usepackage{ifxetex,ifluatex}
\ifnum 0\ifxetex 1\fi\ifluatex 1\fi=0 % if pdftex
  \usepackage[T1]{fontenc}
  \usepackage[utf8]{inputenc}
  \usepackage{textcomp} % provide euro and other symbols
\else % if luatex or xetex
  \usepackage{unicode-math}
  \defaultfontfeatures{Scale=MatchLowercase}
  \defaultfontfeatures[\rmfamily]{Ligatures=TeX,Scale=1}
\fi
% Use upquote if available, for straight quotes in verbatim environments
\IfFileExists{upquote.sty}{\usepackage{upquote}}{}
\IfFileExists{microtype.sty}{% use microtype if available
  \usepackage[]{microtype}
  \UseMicrotypeSet[protrusion]{basicmath} % disable protrusion for tt fonts
}{}
\makeatletter
\@ifundefined{KOMAClassName}{% if non-KOMA class
  \IfFileExists{parskip.sty}{%
    \usepackage{parskip}
  }{% else
    \setlength{\parindent}{0pt}
    \setlength{\parskip}{6pt plus 2pt minus 1pt}}
}{% if KOMA class
  \KOMAoptions{parskip=half}}
\makeatother
\usepackage{xcolor}
\IfFileExists{xurl.sty}{\usepackage{xurl}}{} % add URL line breaks if available
\IfFileExists{bookmark.sty}{\usepackage{bookmark}}{\usepackage{hyperref}}
\hypersetup{
  colorlinks=true,
  linkcolor=black,
  filecolor=Maroon,
  citecolor=green,
  urlcolor=blue,
  pdfcreator={LaTeX via pandoc}}
\urlstyle{same} % disable monospaced font for URLs
\usepackage{listings}
\newcommand{\passthrough}[1]{#1}
\lstset{defaultdialect=[5.3]Lua}
\lstset{defaultdialect=[x86masm]Assembler}
\setlength{\emergencystretch}{3em} % prevent overfull lines
\providecommand{\tightlist}{%
  \setlength{\itemsep}{0pt}\setlength{\parskip}{0pt}}
\setcounter{secnumdepth}{5}

\usepackage[]{biblatex}

\title{Classical Mythology}
\usepackage{etoolbox}
\makeatletter
\providecommand{\subtitle}[1]{% add subtitle to \maketitle
  \apptocmd{\@title}{\par {\large #1 \par}}{}{}
}
\makeatother
\subtitle{Untitled}
\author{William Findlay}
\date{\today}

\begin{document}
\maketitle

\hypertarget{intro-organization}{%
\section{Intro / Organization}\label{intro-organization}}

\hypertarget{sample-question}{%
\subsection{Sample Question}\label{sample-question}}

``What did Agamemnon's murder mean to 5th century Athenians''

\begin{itemize}
\tightlist
\item
  Agamemnon -\textgreater{} Mycenae

  \begin{itemize}
  \tightlist
  \item
    was killed
  \end{itemize}
\item
  many components to this question

  \begin{itemize}
  \tightlist
  \item
    contents of myth
  \item
    history
  \item
    culture
  \item
    interpretation
  \end{itemize}
\end{itemize}

\hypertarget{what-do-lecture-numbers-mean}{%
\subsection{What Do Lecture Numbers
Mean?}\label{what-do-lecture-numbers-mean}}

\begin{itemize}
\tightlist
\item
  week.lecture
\item
  e.g.~3.2

  \begin{itemize}
  \tightlist
  \item
    week 3, lecture 2
  \end{itemize}
\end{itemize}

\hypertarget{grading-and-assignments}{%
\subsection{Grading and Assignments}\label{grading-and-assignments}}

\begin{itemize}
\tightlist
\item
  online quizzes

  \begin{itemize}
  \tightlist
  \item
    open for whole day
  \item
    maximum 8 MC each
  \end{itemize}
\item
  midterm (Monday after the break?)

  \begin{itemize}
  \tightlist
  \item
    in class (but on cuLearn, during class time)
  \item
    part MC
  \item
    part written response (we will know the question a week beforehand)
  \end{itemize}
\item
  either final exam like midterm or final assignment where we re-write a
  story in a given style and justify it
\end{itemize}

\hypertarget{what-is-myth}{%
\section{What is Myth?}\label{what-is-myth}}

\begin{itemize}
\tightlist
\item
  story passed on
\item
  folklore

  \begin{itemize}
  \tightlist
  \item
    what does folk mean here?
  \item
    implicit opposition to writing
  \item
    high culture vs low culture (folk is low) -\textgreater{} implies a
    register
  \end{itemize}
\item
  old wive's tales

  \begin{itemize}
  \tightlist
  \item
    who are old wives? -\textgreater{} metaphor for socio-demographic
    standing
  \end{itemize}
\end{itemize}

\hypertarget{who-is-in-mythological-stories}{%
\subsection{Who is in mythological
stories?}\label{who-is-in-mythological-stories}}

\begin{itemize}
\tightlist
\item
  gods and heroes
\item
  e.g.~Trojan War

  \begin{itemize}
  \tightlist
  \item
    Odysseus, Agamemnon, Zeus, Poseidon, etc.
  \end{itemize}
\item
  gods and heroes -\textgreater{} what contemporary stories might fit
  this definition?

  \begin{itemize}
  \tightlist
  \item
    maybe the Book of Mormon w/ Joseph Smith?
  \end{itemize}
\end{itemize}

\hypertarget{what-happens-in-mythological-stories}{%
\subsection{What happens in mythological
stories?}\label{what-happens-in-mythological-stories}}

\begin{itemize}
\tightlist
\item
  Cadmus and the Dragon

  \begin{itemize}
  \tightlist
  \item
    kills dragon, takes its teeth, sows its teeth
  \item
    teeth grow into men
  \item
    throws a rock, men fight each other for it and kill each other
  \item
    5 men left, become nobility, found city of Thebes with Cadmus
  \end{itemize}
\item
  Eleusinian mysteries (religious cult)

  \begin{itemize}
  \tightlist
  \item
    Heracles
  \item
    Persephone

    \begin{itemize}
    \tightlist
    \item
      daughter of Demeter
    \end{itemize}
  \item
    a priest
  \item
    Demeter (goddess of fertile Earth)

    \begin{itemize}
    \tightlist
    \item
      power of Earth bring forth life, plant life, animal life, human
      life
    \end{itemize}
  \item
    Triptolemus
  \item
    Hecate (titan goddess of witchcraft, necromancy, ghosts, sorcery)
  \item
    Iambe
  \item
    Dionysus (god of wine and revelry)
  \end{itemize}
\item
  rituals

  \begin{itemize}
  \tightlist
  \item
    often explained by stories
  \item
    myths help explain things
  \end{itemize}
\item
  time frame for myths?

  \begin{itemize}
  \tightlist
  \item
    the floating gap
  \item
    sometime in the past
  \item
    purposeful lack of specificity
  \item
    consider:

    \begin{itemize}
    \tightlist
    \item
      Star Wars -\textgreater{} ``a long time ago in a galaxy far, far
      away''

      \begin{itemize}
      \tightlist
      \item
        George Lucas wanted his stories to have a mythological sense to
        them
      \end{itemize}
    \item
      fairy tales -\textgreater{} ``once upon a time''
    \end{itemize}
  \end{itemize}
\end{itemize}

\hypertarget{why-did-greeksromans-care-about-myths}{%
\subsection{Why did Greeks/Romans care about
myths?}\label{why-did-greeksromans-care-about-myths}}

\begin{itemize}
\tightlist
\item
  relationship between past and present
\item
  founding of things that are commonplace in present

  \begin{itemize}
  \tightlist
  \item
    Cadmus -\textgreater{} Thebes
  \item
    Demeter goes to Eleusis -\textgreater{} founds Mysteries
  \end{itemize}
\item
  in general mythes want to:

  \begin{itemize}
  \tightlist
  \item
    justify
  \item
    explain (aetiological)
  \end{itemize}
\item
  key term: \textbf{aetiology}

  \begin{itemize}
  \tightlist
  \item
    causes, reasons, responsibility
  \item
    provides explanation for something
  \end{itemize}
\item
  contemporary myths

  \begin{itemize}
  \tightlist
  \item
    American revolution (second amendment lovers, resist tyranny)
  \item
    storks brining babies
  \end{itemize}
\end{itemize}

\hypertarget{myths-fictions-lies}{%
\subsection{Myths, fictions, lies}\label{myths-fictions-lies}}

\begin{itemize}
\tightlist
\item
  are myths true? do people believe their myths? yes and no
\end{itemize}

\hypertarget{the-no}{%
\subsubsection{The NO}\label{the-no}}

\begin{itemize}
\tightlist
\item
  no one true version
\item
  compare monotheistic religions

  \begin{itemize}
  \tightlist
  \item
    book, word of god, orthodox, heretical
  \item
    there can be variations between sects, but each sect thinks their
    version is absolutely true
  \end{itemize}
\item
  myths contain variations
\item
  e.g., where was Zeus born?

  \begin{itemize}
  \tightlist
  \item
    some say Crete, some say Arcadia
  \end{itemize}
\item
  why was variation okay? mortals can never have certain knowledge about
  gods
\item
  compare ``revealed religion''
\end{itemize}

\hypertarget{the-yes}{%
\subsubsection{The YES}\label{the-yes}}

\begin{itemize}
\tightlist
\item
  myths worked -\textgreater{} they served their purposes

  \begin{itemize}
  \tightlist
  \item
    myth of Cadmus worked in giving a civic identity to Thebans
  \end{itemize}
\end{itemize}

\hypertarget{why-do-we-care-about-myths}{%
\subsection{Why do we care about
myths?}\label{why-do-we-care-about-myths}}

\begin{itemize}
\item
  content vs function
\item
  content

  \begin{itemize}
  \tightlist
  \item
    greek and roman myth are important for understanding references in
    history, later art and culture
  \end{itemize}
\item
  function

  \begin{itemize}
  \tightlist
  \item
    compare how greek and roman myth worked with myths that we see in
    our real lives
  \item
    .e.g., what do these stories accomplish

    \begin{itemize}
    \tightlist
    \item
      Bill Gates = college dropout
    \item
      the Resistance in Star Wars
    \item
      visual myths in advertising (Marlboro man)
    \end{itemize}
  \end{itemize}
\end{itemize}

\hypertarget{provisional-definition-for-myth}{%
\subsection{Provisional definition for
myth}\label{provisional-definition-for-myth}}

\begin{itemize}
\tightlist
\item
  myths are traditional stories embedded within a society that:

  \begin{itemize}
  \tightlist
  \item
    present or encode the way that a society organizes its way of
    thinking about the world
  \end{itemize}
\item
  a way of thinking about the world
\item
  perception and experience about the world
\end{itemize}

\printbibliography[title=Myth and History]

\end{document}
